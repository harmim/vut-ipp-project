% Author: Dominik Harmim <xharmi00@stud.fit.vutbr.cz>

\documentclass[11pt, a4paper]{article}

\usepackage[czech]{babel}
\usepackage[utf8]{inputenc}
\usepackage[T1]{fontenc}
\usepackage{times}
\usepackage[left=2cm, top=3cm, text={17cm, 24cm}]{geometry}
\usepackage[unicode, colorlinks, hypertexnames=false, citecolor=red]{hyperref}

\begin{document}
    {\parindent 0pt \Large
        \textbf{Implementační dokumentace k~1.~úloze do IPP 2018/2019} \\
        Jméno a~příjmení: Dominik Harmim \\
        Login: \texttt{xharmi00}
    }
    
    
    \section{Analýza zdrojového kódu IPPcode19}
    
    Analýza zdrojového kódu IPPcode19 je implementována ve třídě
    \texttt{IPPcode19Parser}. Zdrojový kód je čten po řádcích ze
    standardního vstupu. Jednotlivé řádky jsou testovány PCRE regulárními
    výrazy, čímž jsou kontrolovány lexikální a~syntaktická pravidla jazyka
    IPPcode19. U~každé instrukce se pomocí regulárních výrazů kontrolují počty
    a~typy operandů (proměnná, konstanta, návěští, typ). Hodnoty konstant
    a~použité identifikátory se taktéž analyzují. Před uvedením hlavičky 
    \texttt{.IPPcode19} je možný výskyt prázdných řádků a~komentářů.
    
    Pro generování XML reprezentace zdrojového kódu IPPcode19
    byl použit nástroj \texttt{Array2XML}, který převádí pole v~jazyce PHP
    do formátu XML.
    
    
    \section{Zpracování parametrů příkazové řádky}
    
    Zpracování parametrů příkazové řádky je implementováno ve třídě
    \texttt{ArgumentProcessor}. Je zde implementována metoda \texttt{process},
    která přijímá vstupní argumenty spuštěného skriptu (\texttt{\$argv}) 
    a~na základě přípustných parametrů skriptu tyto argumenty zpracuje
    a~vrací pole zadaných parametrů indexované podle jejich názvu 
    i~s~případnými zadanými hodnotami tak, aby s~nimi bylo dále možné
    jednoduše pracovat. V~případě nepovolených parametrů metoda ukončí
    skript s~návratovým kódem 10.
    
    
    \section{Implementovaná rozšíření}
    
    V~rámci 1.~úlohy bylo implementováno rozšíření pro sbírání 
    statistik\,--\,\textbf{STATP}, viz sekce \ref{sec:statp}.
    
    \subsection{STATP}
    \label{sec:statp}
    
    Pokud je skriptu \texttt{parse.php} předán parametr \texttt{-{}-stats=file},
    tak jsou v~průběhu analýzy sbírány statistiky zpracovávaného zdrojového
    kódu. Na konci analýzy jsou statistiky uloženy do zadaného souboru 
    \texttt{file} v~pořadí, podle dalších upřesňujících parametrů.
    
    Pro každou zaznamenávanou hodnotu je vytvořen číselný čítač, který se 
    v~průběhu analýzy akumuluje. Při zadání parametru \texttt{-{}-loc} se
    počítá počet řádků s~instrukcemi. Při zadání parametru \texttt{-{}-comments}
    se počítá počet komentářů. Při zadání parametru \texttt{-{}-labels} se
    počítá počet definovaných unikátních návěští. A~při zadání parametru
    \texttt{-{}-jumps} se počítá počet instrukcí pro podmíněné a nepodmíněné
    skoky (včetně instrukcí \texttt{CALL} a~\texttt{RETURN}).
    
    Pokud je skriptu \texttt{parse.php} předán parametr \texttt{-{}-stats=file},
    ale nejsou mu předány žádné další parametry, které upřesňují vypisované
    statistiky, tak se do souboru \texttt{file} nic nevypíše. Pokud jsou
    naopak předány nějaké upřesňující parametry, ale není předán parametr
    \texttt{-{}-stats=file}, tak skript skončí s~chybou 10.
\end{document}
